\documentclass[a4paper,10pt]{article}
\usepackage[utf8]{inputenc}
\usepackage{geometry}
\geometry{a4paper, margin=1in}
\usepackage{fontawesome}

\begin{document}

\begin{center}
{\Large \textbf{Souvik Ghosh}} \\
\vspace{0.2cm}
110/22C B.T. Road, Kolkata 700108 \\
bangalmadritista@gmail.com \\
\href{https://www.linkedin.com/in/souvik-ghosh-78aaab232}{\faLinkedin} \hspace{1cm} \href{https://github.com/Souvik-prog}{\faGithub}
\end{center}

\vspace{0.5cm}

\textbf{Personal Details} \\
\begin{itemize}
    \item Father's Name: Tapan Kanti Ghosh
    \item Date of Birth: 28th July 2000
    \item Linguistic Proficiency: English, Bengali, Hindi
    \item Gender: Male
    \item Nationality: Indian
\end{itemize}

\vspace{0.5cm}

\textbf{Areas of Interest} \\
\begin{itemize}
    \item Data Science
    \item Machine Learning and Big Data Analytics
    \item Algorithms
\end{itemize}

\vspace{0.5cm}

\textbf{Software Skills} \\
\begin{itemize}
    \item \textbf{Languages:} C, C++, Java, Python, Data Analysis Using Python
    \item \textbf{Operating Systems:} Windows, Linux, Android
    \item \textbf{Packages:} SCILAB, Pandas, sklearn, scipy
    \item \textbf{DBMS:} Oracle, MySQL, SQLite
\end{itemize}

\vspace{0.5cm}

\textbf{Academic Achievements} \\
\begin{itemize}
    \item Qualified in GATE 2023 AIR - 774
\end{itemize}

\vspace{0.5cm}

\textbf{Education} \\
\begin{tabular}{|l|l|l|l|}
\hline
\textbf{Degree} & \textbf{Year} & \textbf{Institute} & \textbf{Grade/Percentage} \\
\hline
M.Sc & 2023 & Department of Computer Science, Banaras Hindu University, Varanasi & 8.85 \\
\hline
B.Sc (CS) & 2021 & Scottish Church College, University of Calcutta, Kolkata & 8.45 \\
\hline
Class XII & 2018 & Central Modern School & 91.5\% \\
\hline
Class X & 2016 & Central Modern School & 93.2\% \\
\hline
\end{tabular}

\vspace{0.5cm}

\textbf{Project Work} \\
\textit{M.Sc Minor Project (16/03/2023)} \\
\textbf{Title:} Customer Segmentation using K-means Clustering \\
\textbf{Language:} Python \\
\textbf{Libraries:} Pandas, NumPy, Matplotlib, stats, sklearn, scipy \\
\textbf{Algorithm:} K-Means Clustering \\
\textbf{Dataset:} OnlineRetail.xlsx (UCI Machine Learning Repository) \\
\textbf{Objective:}
\begin{itemize}
    \item Identify valuable customers.
    \item Determine customer loyalty.
\end{itemize}
\textbf{Description:} The OnlineRetail.xlsx dataset contains transactions for a UK-based online retail store from December 1st, 2010 to December 9th, 2011. Customers were segmented based on Recency, Frequency, and Total Monetary Spending. Proper utilization of customer segmentation can have a significant positive impact.


\vspace{0.5cm}

\textbf{Declaration} \\
I solemnly declare that all the above information is correct to the best of my knowledge and belief. \\
Saturday, 8th April, 2023

\end{document}